\documentclass[11pt]{article}

\usepackage{hyperref}
\usepackage{subfig}
\usepackage{amsmath}
\usepackage{booktabs}
\usepackage{float}

\title{Discrete optimization, assignment 1}
\author{
    Henrik Bendt --- gwk553 \\
    Jens Fredskov --- chw752 \\
    Jacob Grevald --- xst295
}                                        
\date{October 6, 2014}

\begin{document}
\maketitle

\stepcounter{section} % count the section
\section*{Theoretical part --- formulation and lower bounds}

\subsection{}
First we prove that any feasible TSP solution will satisfy the ILP\@.

We know that a feasible TSP solution is a Hamiltonian path. Any Hamiltonian path
will have exactly 2 edges for every vertex: one going in and one going out. So
for all vertices $v \in \mathcal{V}$ the constraints 
\begin{align}
    \sum_{u\in\mathcal{V},u\neq v} x_{uv} = 1 \label{onein}\\
    \sum_{u\in\mathcal{V},u\neq v} x_{vu} = 1 \label{oneout}
\end{align}
will be satisfied since there will be exactly one edge going into $v$
in our solution (\ref{onein}), and one edge going out of $v$ (\ref{oneout}).

The last constraint says that the number of edges connected to a subset of
vertices of size less than $|\mathcal{V}|-1$ shall be smaller that
$|\mathcal{V}|$. A Hamiltonian path has $|\mathcal{V}|$ edges. When we remove
one or more vertices we remove at least 2 edges. In other words the number of
edges connected to a subset of vertices of size less than $|\mathcal{V}|$ will
be smaller than $|\mathcal{V}-1|$ so the constraint is satisfied.

Finally, the integral constraint is trivially true by the way we encode the
solution: Either and edge is in the solution or it is not in the solution. It is
also trivial to see that the objective function will be minimized when the
Hamiltonian path cost is minimized. \\

Now we prove that a feasible solution to the ILP is a Hamiltonian path. Then it
is trivial to see that the objective function produces the path with minimum
cost.

Constraint (\ref{onein}) and (\ref{oneout}) ensures that each vertex in the
corresponding graph will have degree 2. This means that the choice of edges will
form cycles with no {\it``offshoots''}. They also ensure that each vertex is
included in a cycle.

The last constraint makes sure that there is only one cycle which, combined with
the above, means it will be a Hamiltonian path. 

Consider a cycle of $k\leq|\mathcal{V}|-2$ vertices. The cycle must have $k$
edges, but this conflicts the last constraint so it cannot occur. Now consider a
cycle with $k=|\mathcal{V}-1|$ vertices and edges. This cycle is possible by the
last constraint, but there must be a single vertex left outside the cycle which
is not connected to anything. This conflicts with (\ref{onein}) and
(\ref{oneout}) so it cannot happen. The only possibility left is a single cycle
containing all vertices.


\subsection{}
%1.2
We have $\sum_{k=2}^{n-2} \binom{n}{k} = \sum_{k=2}^{n-2} \frac{n!}{k!(n-k)!}$
combinations of $S \subset V$, such that $2 \leq |S| \leq n-2$ and thus the
total number of these constraints is

\begin{align}
    \sum_{k=2}^{n-2} {\binom{n}{k}}
    &= \sum_{k=0}^{n} {\binom{n}{k}}
        -\binom{n}{0}-\binom{n}{1}-\binom{n}{n-1}-\binom{n}{n} \nonumber \\
    &= 2^n-1-n-n-1 \nonumber \\
    &= 2^n-2n-2
\end{align}

%\begin{align}
%    x2^n-2n-2 
%\end{align}

\subsection{}
%1.3
We have the constraint $t_j \geq t_i + 1 - n(1 - x_{ij})$ for $i\in V, j\in V\backslash \{1\}$. We count the number of constraints in terms of $n = |V|$, we have $n$ assignments of $t_j$ and $n-1$ assignments of $t_j$, thus giving
\begin{align}
    n(n-1)
\end{align}
number of constraints.

\subsection{}
%1.4
%This is too non-specific, but could not find the right slides.
The subtour formulation from exercise 1.1 might be preffered if it
\begin{itemize}
    \item is more effecient at computing tighter lower bounds
    \item has better possibilities of relaxation
    \item has better possibilities of modification of the formulation
\end{itemize}


\subsection{}
%1.5
%Let G be as before and denote one of the vertices as vertex 1. Consider now the subset of edges of G consisting of a tree together with an additional edge incident to vertex 1 such that vertex 1 is in a cycle. Prove that the minimum cost such subset is a lower bound for the minimum cost Hamiltonian tour.
We denote the set $G'$ as all trees of $G$, where each tree has some additional edge incident to a chosen vertex 1 such that vertex 1 is in a cycle. 

A minimum Hamiltonian tour is in the set $G'$, as we simply consider a tree with an additional edge incident to vertex 1 making a minimum tour of $G$.

Since a minimum Hamiltonian tour is in the set $G'$, the minimum of $G'$ must either be a minimum Hamiltonian tour or smaller. Thus a minimum set of edges of $G'$ is a lower bound for a minimum Hamiltonian tour.


\stepcounter{section} % count the section
\section*{Implementation part --- branch-and-bound}

\subsection{}
An upper bound is any feasible solution, that is, any Hamiltonian path. The
first instance is fully connected so we can find a feasible solution by a simple
walk from vertex to vertex. One such walk yields a feasible solution with a cost
of 9.396288. \\

Finding a Hamiltonian path in general is NP-hard, so we cannot rely on this.
Another simple, but bad, upper bound is to find the edge with the highest cost
and multiply the cost by $|\mathcal{V}|$. This provides an upper bound in
%$\mathcal{O}\big(|\mathcal{E}|\big)$ %Capital E is used in the report
$\mathcal{O}\big(|E|\big)$time.

\subsection{}
We have successfully (with a correct solution) run instance 1 with CPLEX, but we have not been able to find a practical way of solving instance 2 and 3 as these are not fully-connected. This means that we have to ascribe a special distance of $\infty$ to all the edges that are not in the graph. This would require e.g.\ a matrix of edges, where an entry is 1 if the edge is in the graph, and $\infty$ if it is not. A matrix like this however would have $76 \times 76 = 5776$ entries which would have to be manually typed in. As this is simply too impractical we did not opt for this solution, and have thus been unable to solve these instances.

\subsection{}
Building upon the given MST algorithm (and correcting the bug) we use 
a minimum 1-tree as lower bound. We have added a field to {\tt Edge.java},
a few fields to {\tt BranchAndBound\_TSP.java} including som initialization in
the constructor and then implemented the lower bound function. We do not utilize
{\tt Kruskal.java}, but used it as a starting point for implementing the lower
bound function.

The statistics of our solution can be found in the table below:
\begin{figure}[H]
    \centering
    \begin{tabular}{llll}
        \toprule
        Instance    & Path length   & Bound nodes   & Running time (ms) \\
        \midrule
        1           & 8.649         & 6832          & 92.32             \\
        2           & 19.030        & 192           & 2.68              \\
        3           & 26.753        & 505378        & 4379.78           \\
        \bottomrule
    \end{tabular}
\end{figure}

\end{document}
