\documentclass[11pt]{article}

\usepackage{hyperref}
\usepackage{subfig}
\usepackage{amsmath}
\usepackage{booktabs}
\usepackage{float}

% Must be loaded after most (read: a shit load incl. hyperref) usepackages.
\usepackage[noabbrev]{cleveref}

\title{Discrete optimization, assignment 2}
\author{
    Henrik Bendt --- gwk553 \\
    Jens Fredskov --- chw752 \\
    Jacob Grevald --- xst295
}                                        
\date{October 30, 2014}

\begin{document}
\maketitle

\stepcounter{section} % count the section
\section*{Theoretical part --- randomized rounding}

We have not been able to solve the theoretical part. We discussed using the same
procedure as Vazirani, but we ran into trouble:

The $\log n$ approximation factor comes from the fact that we take the
union of $c \log n$ independent assignments, but if we remove the
dependence on $n$ then we cannot limit it by $\frac{1}{4n}$ as they do.
This means that when we sum over $\frac{n}{2}$ elements we end up with
a probability for coverage of half the elements that are dependent on
$n$.

On the other hand if the number of independent assignment do depend on $n$ then
we get an approximation factor depending on $n$ when using Markov's inequality.

So we cannot seem to get both using their procedure, and we did not find another
solution to the problem.

\stepcounter{section} % count the section
\section*{Implementation part --- approximation}

\subsection{}
For this part we have used the handed out code to solve the instances. As
predicted by the assignment text the solution can only solve the first three,
and does not seem to terminate (runs for more that 15 minutes
at the very least) on the last. The results of the handed out OPL program can be
seen in \Cref{fig:opl-results}.

\begin{figure}[H]
    \centering
    \begin{tabular}{lll}
        \toprule
        Instance & Objective value & Running time ($\mu$s) \\
        \midrule
        scpa3       & 232     & 85    \\
        scpc3       & 243     & 164   \\
        scpnrf1     & 14      & 2808  \\
        scpnrg5     & ---     & ---   \\
        \bottomrule
    \end{tabular}
    \caption{Exact solutions using CPLEX}\label{fig:opl-results}
\end{figure}

\subsection{}
We have implemented the approximation algorithms Simple Rounding and Random Rounding from Vazirani. The compilation and how to run the program follows from the README file in 2.3.

As we have not completed the theoretical question, we cannot relate it.
Here are the results from the algorithms.
\begin{figure}[H]
    \centering
    \begin{tabular}{lll}
        \toprule
        Instance & Approximated value & Running time ($\mu$s) \\
        \midrule
        scpa3       & 457     & 80    \\
        scpc3       & 587     & 146   \\
        scpnrf1     & 78      & 875   \\
        scpnrg5     & 621     & 600   \\
        \bottomrule
    \end{tabular}
    \caption{Found values and running times for the Simple rounding algorithm on the given instances}\label{fig:simpleround}
\end{figure}

\begin{figure}[H]
    \centering
    \begin{tabular}{lll}
        \toprule
        Instance & Approximated value & Running time ($\mu$s) \\
        \midrule
        scpa3       & 366     & 1818    \\
        scpc3       & 408     & 2291    \\
        scpnrf1     & 25      & 7877    \\
        scpnrg5     & 419     & 11462   \\
        \bottomrule
    \end{tabular}
    \caption{Found values and running times for the Random rounding algorithm on the given instances.}\label{fig:randomround}
\end{figure}


\subsection{}
We have implemented the approximation algorithm from Vazirani that uses the
Primal-dual schema. See the README file for information on compilation and how
to run the program. Our own results are listed in \Cref{fig:c-results}.

\begin{figure}[H]
    \centering
    \begin{tabular}{llll}
        \toprule
        Instance & Approximate cost &
		Total time ($\mu$s) & Algorithm time ($\mu$s) \\
        \midrule
        scpa3       & 448     & ~2100     & ~70    \\
        scpc3       & 463     & ~3700     & ~120   \\
        scpnrf1     & 43      & ~42000    & ~1000  \\
        scpnrg5     & 412     & ~21000    & ~530   \\
        \bottomrule
    \end{tabular}
    \caption{Times are approximate and are only meant to give an idea of the
	effectiveness and of the scale between total and algorithm time, and scale
	between the different instances.}\label{fig:c-results}
\end{figure}

We measure the total running time, and the running time of the core algorithm.
Almost all of the time is used on
parsing the input, which accounts for the large difference. Note that the
approximate cost of this algorithm depends on the order of the input.

\subsection{}
We choose the primal-dual algorithm from 2.3 as our candidate. It should
terminate much faster than 5 minutes even for very large instances and it gives
quite good results, though an iterative method would utilize the allowed time
and probably be better.

\end{document}
